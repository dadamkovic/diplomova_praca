% !TeX spellcheck = en_US
% LaTeX document class
\documentclass[nominted]{uniza}

%-------------------------------------------------------
%             Abbreviation and term database
%-------------------------------------------------------
% !TeX spellcheck = sk_SK

%----------------------------------------------------------
%						Slovník
%----------------------------------------------------------

\DeclareAcronym{viskozita} {
	short = Viskozita,
	long = {Fyzikálna veličina, miera odporu tekutiny deformovať sa pod vplyvom šmykových (tangenciálnych) napätí. Prejavuje sa vnútorným trením.},
	class = dict
}

\DeclareAcronym{zhlukovanie} {
	short = Zhlukovanie,
	long = {Trieda metód strojového učenia, ktoré v daných dátach hľadajú zhluky.},
	extra = {\begin{subdict}
			\item[Hierarchické zhlukovanie] Metódy zhlukovania, kde rozdelenie do zhlukov má hierarchickú štruktúru.
			\item[Fuzzy c-means zhlukovanie] Verzia algoritmu k-means pre fuzzy zhlukovanie.
		\end{subdict}
	},
	class = dict
}

\DeclareAcronym{triedenie} {
	short = Triedenie,
	long = {Pojmy v slovníku sa automaticky triedia podľa abecedy. \hl{Ale pozor: triedenie sa deje prvého argumentu makra \texttt{DeclareAcronym} -- nie podľa poľa \texttt{short}.}},
	class = dict
}

\DeclareAcronym{slovnik_pojmov} {
	short = Slovník pojmov,
	long = {\hl{Slovník pojmov je nepovinný. Na jeho odstránenie stačí zmazať všetky zadefinované pojmy v súbore modules/abbterms.tex.}},
	class = dict
}

%----------------------------------------------------------
%						Skratky
%----------------------------------------------------------

\DeclareAcronym{MAE} {
	short = MAE,
	long = stredná absolútna chyba (\angl{mean absolute error}),
	class = abbrev
}

\DeclareAcronym{ANN} {
	short = ANN,
	long = umelá neurónová sieť (\angl{artificial neural network}),
	class = abbrev
}

\DeclareAcronym{MLP} {
	short = MLP,
	long = {viacvrstvový perceptrón, viacvrstvová neurónová sieť (\angl{multi-layer perceptron})},
	class = abbrev
}


%-------------------------------------------------------
%            Súbory s bibliografickými informáciami
%-------------------------------------------------------
\addbibresource{bibliography.bib}

%-------------------------------------------------------
%                  Jazykové nastavenia
%-------------------------------------------------------
\usepackage[english]{babel}

%-------------------------------------------------------

%                Informácie o dokumente
%-------------------------------------------------------

\title{Realizing walking for a walking robot using deep reinforcement learning}
\subtitle{*The name is a placeholder until I get official english name}
\author{Bc., Daniel Adamkovič}
\keywords{robotics, deep reinforcement learning, artificial intelligence, simulation}
\keywordsSecLang{robotika, hlboké učenie s odmenou, umelá inteligencia, simulácia}

\keywordsName{Keywords}
\keywordsNameSecLang{Kľúčové slová}

\faculty{Faculty of Electrical Engineering and Information Technology}
\department{Department of Control and Information Systems}

% školiace pracovisko, ak je iné než katedra:
%\supervisorinst{
%	SIEMENS\\
%	Kompetenčné centrum Žilina
%}

\facultyshort{FEIT}
\location{Žilina}

\doctype{Master thesis}
\docid{69 69 420}
\supervisor{Ing. PhD., Michal Gregor}


\academicyear{2020/2021} % akademický rok
\submissionyear{2020} % rok odovzdania práce
% Študijný program:
\studyprogramme{Cybernetics}
% Študijný odbor:
\fieldofstudy{5.2.14 Automation}

% Abstrakt v hlavnom jazyku
\abstract{Abstract}{
Abstrakt obsahuje informáciu o cieľoch práce, jej stručnom obsahu a v závere abstraktu sa charakterizuje splnenie cieľa, výsledky a význam celej práce. Abstrakt sa píše súvisle ako jeden odsek a jeho rozsah je spravidla 100 až 500 slov.
}

% Abstrakt v cudzom jazyku (anglickom, nemeckom, ...)
\abstractSecLang{Abstrakt}{
Nejaký abstrakt po slovensky.
}

\date{Dátum odovzdania práce}

\acknowledgements{
	Poďakovanie nie je povinné. Ak nemá byť zahrnuté, stačí túto časť zakomentovať.
}

%-------------------------------------------------------
%		 Vybrané metadáta zapíšeme aj do dokumentu.
%-------------------------------------------------------

\hypersetup{
	pdfauthor={\Author},%
    pdftitle={\Title},%
    pdfsubject={\Doctype},%
    pdfkeywords={\Keywords},%
    pdfproducer={LaTeX},%
%    pdfcreator={pdfLaTeX}
}

%-------------------------------------------------------
%						Includeonly
%-------------------------------------------------------

%\includeonly{
%kap_uvod
%}

%-------------------------------------------------------
%		Korektné zalamovanie spojok na konci riadku.
%-------------------------------------------------------
%\usepackage{encxvlna}
% a temporary workaround for encxvlna breaking multiarg macros


%-------------------------------------------------------
%					Začiatok dokumentu
%-------------------------------------------------------

\begin{document}

%-------------------------------------------------------
%				 Obálka a titulná strana
%-------------------------------------------------------

\makecover
\maketitle

%-------------------------------------------------------
%						 Zadanie
%-------------------------------------------------------

\includepdf[fitpaper]{modules/zadanie.pdf}

%-------------------------------------------------------
%				Front Matter (TOC, LOF, ...)
%-------------------------------------------------------
\frontmatter

% suppress some commands in TOC and lists
\begingroup
\renewcommand{\ac}[1]{#1}
\renewcommand{\cite}[1]{}

% Poďakovanie
\makeacknowledgements

% Abstrakt, anotácia
\makeabstract

%  TOC
\tableofcontents

% list of figures
\iftotalfigures\listoffigures\fi

% list of tables
\iftotaltables\listoftables\fi

% list of abbreviations
\acsetup{page-ref=comma,list-style=acronyms
}{
	\ifargoutempty{\printacronyms[include-classes=abbrev,heading=none]}{}{
		\unchapter{Abbreviations}
		\argoutemptyFirstArg
	}
}

% dictionary
\acsetup{page-ref=none,list-style=dictstyle,extra-style=plain,extra-format={\;},only-used=false}{
\ifargoutempty{\printacronyms[include-classes=dict,heading=none]}{}{
	\unchapter{Dictionary of terms}
	\setlength{\columnseprule}{0.2pt}
	\setlength{\columnsep}{1.25cm}
	\begin{multicols}{2}
	\argoutemptyFirstArg
	\end{multicols}
}}

\endgroup % suppress some commands in TOC and lists

%-------------------------------------------------------
%						Document
%-------------------------------------------------------
\mainmatter

\include{chapter_0}
% !TeX spellcheck = en_US
\chapter{Deep reinforcement learning}
In this chapter we will put \ac{DRL} into the context of broader field of artificial intelligence and explain how it differs from other approaches that attempt to infuse agents with some form of intelligence. Afterwards we will proceed to introduce the main algorithms that are relevant for the use in environments with continuous action spaces.  
\section{DRL in the context of AI}



%-------------------------------------------------------
%                     Bibliography
%-------------------------------------------------------

\printbibliography[heading=unchapter,title={Zoznam použitej literatúry}]

%-------------------------------------------------------
%					Čestné vyhlásenie
%-------------------------------------------------------

\makeDeclaration

%-------------------------------------------------------
%						Appendix
%-------------------------------------------------------

\makeAppendixPage
\appendix

% !TeX spellcheck = sk_SK

\chapter{Zeleninový šalát}

Tvorba príloh je veľmi jednoduchá -- stačí ich pridať ako nové kapitoly v časti dokumentu označenej ako \texttt{{\textbackslash}appendix}. Prílohy sú automaticky číslované nie numericky, ale písmenami abecedy, čím sú dostatočne odlíšené od klasických kapitol. Ak práca obsahuje prílohy, šablóna automaticky vygeneruje titulný list oddeľujúci prílohovú časť práce od hlavnej časti práce.

\section{Textová vata}

\Blindtext


\end{document}
